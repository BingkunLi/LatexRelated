\documentclass[12pt,a4paper]{book} %\chapter / \section / \subsection 三级
% 导言区,加载宏包和各项设置
\usepackage{ctex} % 中文支持
\usepackage{import} % 每部分编译文件只在各子文件夹中寻找
\usepackage{standalone} %无视imput和import文件中的宏包与环境
%\usepackage[utf8]{inputenc} %非英语、非中文的欧洲语言排版
\usepackage{amsmath} %包含数学公式
\usepackage{amssymb} %数学字体与符号,必备。注意此宏包已经包括了 amsfonts,不要重复使用
\newtheorem{function}{Formula}
\usepackage{graphicx} %插入图片
\usepackage{natbib} % Use natbib to manage the reference % 此处示意对参考文献和索引的设置
\usepackage{makeidx} %在最后做索引
\usepackage{xcolor} % 带上颜色
\usepackage{titlesec} %设置分级section样式
\titleformat{\subsection}[block]{\color{blue}\Large\bfseries\filcenter}{}{1em}{}
\makeindex
\bibliographystyle{apalike}
\begin{document}
\frontmatter % 小写罗马字母页码,其后chapter不编号
\title{\LARGE{\emph{\textbf{\kaishu 研究生期间笔记}}}} \author{李某\thanks{E-mail~:~bk$\_$li@139.com}\and 爱他和他爱的人们\thanks{家人、朋友}} \date{始于2017年06月29日,一直在路上}
\maketitle % 标题页 可使用\titlepage命令构建环境
% 前言章节preface.tex
\import{notedetails/}{preface} 
 % 目录,build twice
\tableofcontents
 % 以下是正文部分,页码为阿拉伯数字,其后章节编号正常
\mainmatter
\chapter{组会}
\indent 研一上学期每周四14:30——17:00
\section{研一上学期}
\import{seminar/}{SongLabSeminarFirstUp}
\section{研一下学期}
%...
\chapter{实验记录}
\indent 李某经历或接手的实验与记录
\section{Cue $\alpha$与N2pc、CDA的关系实验——条件一}
\indent 从2018年4月24日开始
\import{ExperimentRecording/}{NewCueCon1}
\chapter{研究生课堂笔记}
\section{神经科学基础}
\indent 研一上学期每周一18:00——19:40
\import{course/}{神经科学基础}
\section{心理学研究方法}
\indent 研一上学期每周五8:00——10:45
\import{course/}{心理学研究方法}
\section{动力系统分析}
\indent 研一上学期每周五13:30——15:10
\import{course/}{动力系统分析}
\section{认知神经科学}
\indent 研一上学期每周二13:30——15:15
\import{course/}{认知神经科学}
\section{神经信号处理}
\indent 研一下学期每周四18:00——20:45
\import{course/}{神经信号处理}
\chapter{讲座}
\indent 研究生期间所参加的讲座
\import{course/}{lecture}
% 附录
\chapter{兴趣研究}
\section{\LaTeX}
\indent 该工具用于演示文稿、笔记、参考文献、排版
\import{interests/}{latexnote}
\section{EEGLAB}
\indent 该工具用于脑电数据的分析
\import{interests/}{EEGLABnote}
\section{ERP}
\indent 阅读接触的ERP成分
\import{interests/}{ERPnote}
\appendix
 % 后记部分,页码格式不变,其后章节不编号
\backmatter
% 后记prologue.tex
% 利用BibTeX 工具生成参考文献,其内容为路径/bib文件名
%\bibliography{...} 
\printindex % 利用makeindex 工具生成索引
\end{document}